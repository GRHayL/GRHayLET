\documentclass{article}

% Use the Cactus ThornGuide style file
% (Automatically used from Cactus distribution, if you have a 
%  thorn without the Cactus Flesh download this from the Cactus
%  homepage at www.cactuscode.org)
\usepackage{../../../../doc/latex/cactus}
\usepackage{xspace}
\newcommand{\grhayl}{\texttt{GRHayL}\xspace}
\newcommand{\glib}{\texttt{GRHayLib}\xspace}
\newcommand{\ghd}{\texttt{GRHayLHD}\xspace}
\newcommand{\igm}{\texttt{IllinoisGRMHD}\xspace}
\newcommand{\hydrobase}{\texttt{HydroBase}\xspace}

\begin{document}

\title{IllinoisGRMHD}
\author{Samuel Cupp \\ Leonardo Rosa Werneck \\ Terrence Pierre Jacques \\ Zachariah Etienne}
\date{$ $Date$ $}

\maketitle

% Do not delete next line
% START CACTUS THORNGUIDE

\begin{abstract}
\igm solves the equations of general relativistic
magneto-hydrodynamics (GRMHD) using the General Relativistic
Hydrodynamics Library (\grhayl), which is included in the
Einstein Toolkit via the \glib thorn. The core library
inherits or adapts code from the original \igm thorn into a
modular, infrastructure-agnostic library. This original thorn
is itself a rewrite of the Illinois Numerical Relativity (ILNR)
group's GRMHD  code (ca. 2009--2014), which was used in their
modeling of the following systems:
\begin{enumerate}
\item Magnetized circumbinary disk accretion onto binary black holes
\item Magnetized black hole--neutron star mergers
\item Magnetized Bondi flow, Bondi-Hoyle-Littleton accretion
\item White dwarf--neutron star mergers
\end{enumerate}

The improved \grhayl-based \igm incorporates more options
for controlling the evolution, most notably the addition
of new conservative-to-primitive solvers and nuclear equations of
state.
\end{abstract}

\section{Introduction}

This thorn provides a GRMHD evolution code built on
the General Relativistic Hydrodynamics Library (\grhayl),
which is included in the Einstein Toolkit via the \glib
thorn. The core library inherits or adapts code from the
original \igm thorn into a modular, infrastructure-agnostic
library.

The new and improved \igm provides support for hybrid and
tabulated EOS, both with and without entropy. The core functions
all come from \glib, ensuring that any improvements or updates
to the library can be easily adopted by the thorn. The
conservative-to-primitive solver is selected at runtime, allowing
for users to take advantage of any Con2Prim methods provided by \grhayl.

\section{Parameters}

Most of the behavior of this thorn is controlled by \glib,
the thorn that provides \grhayl functionality
within the Einstein Toolkit. However, there are several
parameters in the thorn which control diagnostic or debugging
features. Most are self-explanatory, but some parameters deserve
a more detailed explanation.

First, many analysis or diagnostic thorns use \hydrobase variables.
By default, \igm never copies the data back from \igm variables
to \hydrobase variables, which would prevent the usage of these
thorns. The \textbf{Convert\_to\_HydroBase\_every} parameter
sets how often to copy back this data. Naturally, copying more
frequently will slow down the simulation, so this should be set
on the same frequency as the analysis or IO thorn that is using
the HydroBase variable. We reduced the amount of data copying in \igm
by using the HydroBase variables directly, which removes the need
to copy them back and removes extra grid functions from memory.
We still use distinct velocity variables because we use a different
velocity than is defined in HydroBase, so this conversion is still
necessary.

Next, there are several parameters for debugging or other testing.
The perturbing parameters give users the ability to affect either
the initial data or evolution data at runtime without slowing down
standard simulations. The \textbf{perturb\_initial\_data} parameter
controls whether the initial data from HydroBase is perturbed, and
the \textbf{perturb\_every\_con2prim} parameter controls whether
the conservative variables are perturbed before every con2prim call.
The magnitude of these perturbations is set by \textbf{random\_pert}.

\section{Updating Old Parfiles}

With the many changes to improve the code and transition to using \glib,
parfile setups have changed significantly. To this end, we provide a
guide to updating parfiles from old \igm to the \grhayl-based \igm.
Old parfiles are temporarily supported via backward compatibility
patch, but this support ends in the ET\_2024\_11 release.

First, the two thorns ID\_converter\_ILGRMHD and Convert\_to\_HydroBase
are entirely deprecated and will be removed. As such, these thorns and
all their parameters should be removed from the parfile. Their equivalent
parameters are as follows:
%
\begin{center}
\begin{tabular}{|l l|}
 \multicolumn{1}{|c}{Old} & \multicolumn{1}{c|}{New} \\\hline
 ID\_converter\_ILGRMHD::Gamma\_Initial & GRHayLib::Gamma\_ppoly\_in[0] \\ 
 ID\_converter\_ILGRMHD::random\_seed   & IllinoisGRMHD::random\_seed \\  
 ID\_converter\_ILGRMHD::random\_pert   & IllinoisGRMHD::random\_pert \\
 ID\_converter\_ILGRMHD::K\_Initial     & GRHayLib::k\_ppoly0 \\
 Convert\_to\_HydroBase::Convert\_to\_HydroBase\_every & IllinoisGRMHD::Convert\_to\_HydroBase\_every \\\hline
\end{tabular}
\end{center}
%
The parameter ID\_converter\_ILGRMHD::pure\_hydro\_run is entirely
deprecated, as the \ghd thorn provides this functionaility. Additionally,
to properly trigger the random perturbation the new parameter
IllinoisGRMHD::perturb\_initial\_data should be set.

The parameters in \igm have also changed significantly, with many
moving to the \glib thorn. These changes are as follows:
%
\begin{center}
\begin{tabular}{|l l|}
 \multicolumn{1}{|c}{Old} & \multicolumn{1}{c|}{New} \\\hline
 IllinoisGRMHD::GAMMA\_SPEED\_LIMIT & GRHayLib::max\_Lorentz\_factor \\
 IllinoisGRMHD::K\_poly             & GRHayLib::k\_ppoly0 \\
 IllinoisGRMHD::rho\_b\_atm         & GRHayLib::rho\_b\_atm \\
 IllinoisGRMHD::rho\_b\_max         & GRHayLib::rho\_b\_max \\
 IllinoisGRMHD::Psi6threshold       & GRHayLib::Psi6threshold \\
 IllinoisGRMHD::neos                & GRHayLib::neos \\
 IllinoisGRMHD::gamma\_th           & GRHayLib::Gamma\_th \\
 IllinoisGRMHD::damp\_lorenz        & GRHayLib::Lorenz\_damping\_factor \\\hline
\end{tabular}
\end{center}
%
As seen here, the ID converter and \igm thorns both had a $K$ parameter.
These are condensed into a single parameter in GRHayLib.

Two parameters are deprecated and marked for removal. These are
IllinoisGRMHD::tau\_atm and IllinoisGRMHD::conserv\_to\_prims\_debug. The former
is now automatically computed by \glib, and the latter's feature is no
longer available.

Other \igm parameter names are unchanged. However, some of their options
are deprecated and also marked for removal. These are the "essential" and
"essential+iteration output" options for IllinoisGRMHD::verbose.

Finally, note that the magnetic field quantities have been rescaled by
$\left(4\pi\right)^{-1/2}$, as this simplifies many equations in \grhayl.
This definition is consistent with the magnetic field defined by HydroBase
and GRHydro (which the old \igm did not respect). To maintain support for
old initial data thorns, we by default assume that the HydroBase $B^i$,
$A_i$ use the old definition and rescale them when copying into the \igm
variables. The parameter rescale\_magnetics controls this behavior. If
the thorns using the old definition (i.e. thorns in wvuthorns and
wvuthorns\_diagnostics) are updated to use the new definition, this parameter
will be switch such that the default behaves as expected for these thorns.
Such a change will be detailed in the Einstein Toolkit release notes prior
to the release introducing the change.

In addition to these changes, \glib offers new parameter options for
controlling the simulation, such as a wider range of conservative-to-primitive
routines, new equation of state options, and control over the PPM reconstruction
parameters. Please see that thorn's documentation for more information about these
features, as well as \grhayl's documentation on the library's implementation
details.

\section{Acknowledgements}

Note that many codes have fed into the \grhayl library. The most prominent is
the \igm thorn developed primarily by Zachariah Etienne, Yuk Tung Liu, and
Vasileios Paschalidis~\cite{WVUThorns_IllinoisGRMHD_Etienne:2015cea}. This
thorn is based on the GRMHD code of the Illinois Numerical Relativity group
(ca. 2014), written by Matt Duez, Yuk Tung Liu, and Branson Stephens~.

For the full list of contributions and related citations, please use \grhayl's
\href{https://github.com/GRHayL/GRHayL/wiki/Citation-and-License-Guide}{citation guide}.

\begin{thebibliography}{9}

\bibitem{WVUThorns_IllinoisGRMHD_Etienne:2015cea}
Z.~B.~Etienne, V.~Paschalidis, R.~Haas, P.~M\"osta and S.~L.~Shapiro,
``IllinoisGRMHD: An Open-Source, User-Friendly GRMHD Code for Dynamical
Spacetimes,''
Class. Quant. Grav. \textbf{32}, 175009 (2015)
doi:10.1088/0264-9381/32/17/175009
[\href{https://arxiv.org/abs/1501.07276}{arXiv:1501.07276} [astro-ph.HE]].

\end{thebibliography}

% Do not delete next line
% END CACTUS THORNGUIDE

\end{document}
