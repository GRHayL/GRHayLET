\documentclass{article}

% Use the Cactus ThornGuide style file
% (Automatically used from Cactus distribution, if you have a 
%  thorn without the Cactus Flesh download this from the Cactus
%  homepage at www.cactuscode.org)
\usepackage{../../../../doc/latex/cactus}
\usepackage{xspace}
\newcommand{\igm}{\texttt{IllinoisGRMHD}\xspace}
\newcommand{\hydrobase}{\texttt{HydroBase}\xspace}

\begin{document}

\title{IllinoisGRMHD}
\author{Samuel Cupp \\ Terrence Pierre Jacques \\ Leonardo Rosa Werneck \\ Zachariah Etienne}
\date{$ $Date$ $}

\maketitle

% Do not delete next line
% START CACTUS THORNGUIDE

\begin{abstract}
Provides a general relativistic magnetohydrodynamic (GRMHD)
evolution code based on the code from the Illinois group.
This updated version of the \igm thorn uses GRHayL via the
GRHayLib thorn.
\end{abstract}

\section{Introduction}

This thorn provides a GRMHD evolution code built on
the General Relativistic Hydrodynamics Library (GRHayL).
The core library inherits or adapts code from the original
\igm thorn into a modular, infrastructure-agnostic
library. This updated version of \igm provides a wider
range of con2prim and EOS options thanks to the underlying
improvements in GRHayL.

\section{Parameters}

Most of the behavior of this thorn is controlled by
\texttt{GRHayLib}, the thorn that provides GRHayL functionality
within the Einstein Toolkit. However, there are several
parameters in the thorn which control diagnostic or debugging
features. Most are self-explanatory, but some parameters deserve
a more detailed explanation.

First, many analysis or diagnostic thorns use \hydrobase variables.
By default, \igm never copies the data back from \igm variables
to \hydrobase variables, which would prevent the usage of these
thorns. The \textbf{Convert\_to\_HydroBase\_every} parameter
sets how often to copy back this data. Naturally, copying more
frequently will slow down the simulation, so this should be set
on the same frequency as the analysis or IO thorn that is using
the HydroBase variable.

Next, there are several parameters for debugging or other testing.
The perturbing parameters give users the ability to affect either
the initial data or evolution data at runtime without slowing down
standard simulations. The \textbf{perturb\_initial\_data} parameter
controls whether the initial data from HydroBase is perturbed, and
the \text{perturb\_every\_con2prim} parameter controls whether
the conservative variables are perturbed before every con2prim call.
The magnitude of these perturbations is set by \textbf{random\_pert}.

% Do not delete next line
% END CACTUS THORNGUIDE

\end{document}
