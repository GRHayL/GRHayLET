\documentclass{article}

% Use the Cactus ThornGuide style file
% (Automatically used from Cactus distribution, if you have a 
%  thorn without the Cactus Flesh download this from the Cactus
%  homepage at www.cactuscode.org)
\usepackage{../../../../doc/latex/cactus}
\usepackage{xspace}
\newcommand{\grhayl}{\texttt{GRHayL}\xspace}
\newcommand{\glib}{\texttt{GRHayLib}\xspace}
\newcommand{\ghd}{\texttt{GRHayLHD}\xspace}
\newcommand{\igm}{\texttt{IllinoisGRMHD}\xspace}
\newcommand{\hydrobase}{\texttt{HydroBase}\xspace}

\begin{document}

\title{GRHayLHD}
\author{Samuel Cupp \\ Leonardo Rosa Werneck \\ Terrence Pierre Jacques \\ Zachariah Etienne}
\date{$ $Date$ $}

\maketitle

% Do not delete next line
% START CACTUS THORNGUIDE

\begin{abstract}
Provides a general relativistic hydrodynamic (GRHD) evolution code
using \grhayl via the \glib thorn. This is a trimmed-down version
of the \igm thorn where magnetic fields are set to zero.
\end{abstract}

\section{Introduction}

This thorn provides a GRHD evolution code built on
the General Relativistic Hydrodynamics Library (\grhayl),
which is included in the Einstein Toolkit via the \glib
thorn. The core library inherits or adapts code from the
original \igm thorn into a modular, infrastructure-agnostic
library. While the \igm thorn can be set to have
zero magnetic fields, they will still be evolved. This
is much slower than not evolving them, and \ghd provides an
implementation for evolving systems without magnetic
fields.

This thorn provides support for hybrid and tabulated EOS, both
with and without entropy. The core functions all come from \glib,
ensuring that any improvements or updates to the library can
be easily adopted by the thorn. The conservative-to-primitive
solver is selected at runtime, allowing for users to take
advantage of any Con2Prim methods provided by \grhayl.

\section{Parameters}

Most of the behavior of this thorn is controlled by \glib,
the thorn that provides \grhayl functionality
within the Einstein Toolkit. However, there are several
parameters in the thorn which control diagnostic or debugging
features. Most are self-explanatory, but some parameters deserve
a more detailed explanation.

First, many analysis or diagnostic thorns use \hydrobase variables.
By default, \ghd never copies the data back from \ghd variables
to \hydrobase variables, which would prevent the usage of these
thorns. The \textbf{Convert\_to\_HydroBase\_every} parameter
sets how often to copy back this data. Naturally, copying more
frequently will slow down the simulation, so this should be set
on the same frequency as the analysis or IO thorn that is using
the HydroBase variable. We reduced the amount of data copying in \ghd
by using the HydroBase variables directly, which removes the need
to copy them back and removes extra grid functions from memory.
We still use distinct velocity variables because we use a different
velocity than is defined in HydroBase, so this conversion is still
necessary.

Next, there are several parameters for debugging or other testing.
The perturbing parameters give users the ability to affect either
the initial data or evolution data at runtime without slowing down
standard simulations. The \textbf{perturb\_initial\_data} parameter
controls whether the initial data from HydroBase is perturbed, and
the \textbf{perturb\_every\_con2prim} parameter controls whether
the conservative variables are perturbed before every con2prim call.
The magnitude of these perturbations is set by \textbf{random\_pert}.

% Do not delete next line
% END CACTUS THORNGUIDE

\end{document}
