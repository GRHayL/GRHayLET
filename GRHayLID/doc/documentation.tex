\documentclass{article}

% Use the Cactus ThornGuide style file
% (Automatically used from Cactus distribution, if you have a 
%  thorn without the Cactus Flesh download this from the Cactus
%  homepage at www.cactuscode.org)
\usepackage{../../../../doc/latex/cactus}
\usepackage{xspace}
\newcommand{\gid}{\texttt{GRHayLID}\xspace}
\newcommand{\igm}{\texttt{IllinoisGRMHD}\xspace}
\newcommand{\hydrobase}{\texttt{HydroBase}\xspace}

\begin{document}

\title{GRHayLHD}
\author{Samuel Cupp \\ Terrence Pierre Jacques \\ Leonardo Rosa Werneck \\ Zachariah Etienne}
\date{$ $Date$ $}

\maketitle

% Do not delete next line
% START CACTUS THORNGUIDE

\begin{abstract}
Provides simple magnetohydrodynamic (MHD) initial data for testing.
This includes the Balsara tests, a cylindrical explosion, TODO. While
some initial data requires EOS information provided by \texttt{GRHayLib},
the data can then be used by any evolution thorn so long as their EOS
framework is set up to match \texttt{GRHayLib}.
\end{abstract}

\section{Introduction}

This thorn provides a variety of initial data setups for
testing GRMHD evolution codes. For 1D, this includes all five
Balsara tests, equilibrium initial data, a sound wave, and
a shock tube. TODO 2D. For 3D, we provide isotropic gas and
constant-density sphere initial data.

The initial data is separated into magnetic and hydrodynamic
parts, so the magnetic initial data can be turned off if desired.
All settings use the \hydrobase variables for compatibility with
other thorns, but only Avec is set. The Aphi or B variables are
not set by this thorn.

\section{Parameters}

The parameters are mostly self-explanatory or just extensions of
\hydrobase keywords. The param.ccl is divided into sections based
on which tests the parameters are for, but we explain them here.
The type of test is controlled by \textbf{initial\_hydro} from
\hydrobase. All the options for these parameters are explicitly
listed by the autogenerated parameter documentation below.

\subsection{HydroTest1D}

This option is used for all the 1D tests we provide. The
\textbf{test\_1D\_initial\_data} parameter chooses the
specific test and defaults to the Balsara1 test. For shocks and
waves, the \textbf{test\_shock\_direction} parameter changes the
direction of the shock, and the \textbf{discontinuity\_position}
parameter changes the location of the shock front. For the sound
wave test, the amplitude can be set with \textbf{test\_wave\_amplitude}.

\subsection{IsotropicGas}

This option sets the hydrodynamic variables to an isotropic gas using
the tabulated equation of state and does not work with other EOS
settings. It does not set magnetic quantities. The thorn checks to
ensure parameters are set correctly in \texttt{GRHayLib} and
\texttt{HydroBase} so that the initial data works, and the run will end
with an error message if the core parameters are misconfigured.

Specifically, the thorn checks the following parameters:
\begin{tabular}{l|l}
Thorn \& Parameter & Correct Setting \\\hline
\texttt{GRHayLib::EOS\_type} & "tabulated" \\
\texttt{HydroBase::initial\_Y\_e} & "GRHayLID" \\
\texttt{HydroBase::initial\_temperature} & "GRHayLID"
\end{tabular}

\subsection{ConstantDensitySphere}

This option sets the hydrodynamic variables to a sphere of constant
density using the tabulated equation of state and does not work
with other EOS settings. It does not set magnetic quantities. The thorn
checks to ensure parameters are set correctly in \texttt{GRHayLib} and
\texttt{HydroBase} so that the initial data works, and the run will end
with an error message if the core parameters are misconfigured.

Specifically, the thorn checks the following parameters:
\begin{tabular}{l|l}
Thorn \& Parameter & Correct Setting \\\hline
\texttt{GRHayLib::EOS\_type} & "tabulated" \\
\texttt{HydroBase::initial\_Y\_e} & "GRHayLID" \\
\texttt{HydroBase::initial\_temperature} & "GRHayLID"
\end{tabular}

% Do not delete next line
% END CACTUS THORNGUIDE

\end{document}
